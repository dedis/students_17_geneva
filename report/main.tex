\documentclass[]{article}
\usepackage{graphicx}
%\graphicspath{ {images/} }
\usepackage{wrapfig}
\usepackage{enumerate}
\usepackage{natbib}
\usepackage{url}
\usepackage{amsfonts}
\usepackage{mathtools}
\usepackage{subcaption}
\usepackage{afterpage}

\newcommand{\ZZ}{\ensuremath{\mathbb{Z}}}
%opening
\title{CHVote: Efficient modular exponentiations}
\author{Nicolas GAILLY}

\begin{document}

\maketitle

\section{Introduction} \label{intro}

The canton of Geneva is working on a new digital votation system called CHVote
\cite{chvote} whose main goal is to increase the electorate confidence by
providing a secure, transparent and usable online voting system. The final
product must enable a Swiss citizen to vote on any election mandated by the
Geneva canton using his laptop or mobile phone in a browser environment.

In this work, we focus on the vote casting part of the system from the client
side. In order to cast a vote, the client has to perform an k-out-of-n oblivious
transfer protocol \cite{chu2005} with the votation servers. In this protocol,
the client must perform between a few and a hundred of modular exponentiation
computation, depending on the number of votes the client has to perform for a
particular votation. In the context of CHVote, these modular exponentiation
computations take place in a multiplicative group whose order is a large prime
$q$. It is typically expected for security reason in this scenario that the number
of bits needed to represent q can lie between 1024 bits and up to 8192 bits.

The potentially large number of computation on the client side is problematic in
this context.  Indeed, modular exponentiation with a large modulo is a
computationally expensive operation. Typical optimizations in this space
consists of hand-written assembly code such as the GMP library \cite{gmplib}
which is composed of a mix of C and assembly. Typically, any developer that wishes
to perform modulor exponentiation has to write the code in native Javascript or
use a Javascript library. Javascript being an interpreted language running in
the browser context performs quite poorly in comparison with native C/asm
code~\cite{jsbad}.

In this work, we present an efficient way of computing these modular
exponentations relying on multiple third party servers and evaluate this
solution against the state-of-the-art methods available in the browser. This
work presents our solution in section \ref{design}, our evaluation in section
\ref{evaluation}, and finally some open questions to be addressed in section
\ref{discussions}.

\section{Design} \label{design}

The goal of the project is to find an efficient scheme for the client to perform
modular exponentiations. In the context of CHVote, the client only must know
about the exponent; the base and the modulo need not to be private, simplifying
significantly the design of the solution.

The core idea is to decentralize the expensive computation to third party
servers while keeping the exponent secret by using a trivial non-threshold
secret sharing scheme. The client computes the final result from the individual
computations by using inexpensive modular \textit{multiplication}
operations.These servers are modeled as honest-but-curious adversaries and
should ideally be independent and highly available.


We assume we are working in a multiplicative group $G$ of order prime $q$. We denote
the base $b \in \ZZ_q^*$ and the exponent $a \in \ZZ_q^*$. We want to find an
efficient way to compute: $$ b^a \mod q $$

We call $S = \{s_1,\dots,s_n\}$ the set of $n$ servers selected to help
the client with his computation. The protocol works as follow:

\begin{enumerate}

    \item The client creates the $n$ shares:

        \begin{gather}
            \nonumber r_i = \text{\textit{ random }} \in \ZZ_q^* \text{  for } i=0...n-2\\
            \nonumber r_{n-1} = {a - \sum_{i=0}^{n-2} r_i} \mod q
        \end{gather}

    \item The client sends a computation request to each server $s_i$ over a
        secure channel. The request sent to server $i$ contains the share $r_i$,
        the base $b$ and the modulo $q$.

    \item Upon reception of a request, the server simply computes $ v_i = b^r_i
        \mod q$ and sends the response $v_i$ to the client.

    \item The client waits to receive $n$ responses and then computes the final
        result $v$ as: 

        \begin{align}
          \nonumber v &= \prod_{i=0}^{n-1} {v_i \mod q}\\
          \nonumber   &= b^{\sum_{i=0}^{n-1} r_i} \mod q\\
          \nonumber   &= b^{\sum_{i=0}^{n-2} r_i} * b^{r_{n-1}} \mod q\\
          \nonumber   &= b^a \mod q
        \end{align}

\end{enumerate}

This scheme allows an efficient outsourcing of the computation by allowing the
client to use only inexpensive operations.

\section{Evaluation} \label{evaluation}

\begin{figure}[ht!]

\centering

\begin{subfigure}[b]{0.8\textwidth}
   \label{fig:plot1024}
   \includegraphics[width=\textwidth]{plot1024}
   \caption{1024 bits key size}
\end{subfigure}
~
\begin{subfigure}[b]{0.8\textwidth}
   \label{fig:plot2048}
   \includegraphics[width=\textwidth]{plot2048}
   \caption{2048 bits key size}
\end{subfigure}
~
\begin{subfigure}[b]{0.8\textwidth}
   \label{fig:plot4096}
   \includegraphics[width=\textwidth]{plot4096}
   \caption{4096 bits key size}
\end{subfigure}

\caption{\textbf{Modular exponentiation comparison between "local" and the
    "split" method with different key sizes}. The "Split" method is the
    splitting strategy using 3 different remote web servers. The "Local" method
    is doing the computation in-browser with Javascript.}

\label{fig:modexp:js}
    
\end{figure}

\begin{figure}[ht!]

\centering

\begin{subfigure}[b]{0.8\textwidth}
   \label{fig:plot1024_wasm}
   \includegraphics[width=\textwidth]{plot1024}
   \caption{1024 bits key size}
\end{subfigure}
~
\begin{subfigure}[b]{0.8\textwidth}
   \label{fig:plot2048_wasm}
   \includegraphics[width=\textwidth]{plot2048}
   \caption{2048 bits key size}
\end{subfigure}
~
\begin{subfigure}[b]{0.8\textwidth}
   \label{fig:plot4096_wasm}
   \includegraphics[width=\textwidth]{plot4096}
   \caption{4096 bits key size}
\end{subfigure}

\caption{\textbf{Modular exponentiation comparison between "WebAssembly" and the
    "split" method with different key sizes}. The "split" method is the
    splitting strategy using 3 different remote web servers. The "WebAssembly" method
    is doing the computation in-browser but using the WebAssembly technology
    within Javascript.}

\label{fig:modexp:wasm}
    
\end{figure}

In order to be relevant, this proposed approach must be able to outperform the
performance of the computation in a browser. In order to evaluate the
system, we have designed a Javascript benchmark framework enabling different
strategies to be evaluated. The code is available on Github \cite{code}.

The evaluation consists of comparing the time it takes to compute $N$ modular
exponentiations within the browser using Javascript (the "local" method), using
WebAssembly~\cite{haas2017bringing}, a recent effort to be able to run
applications at a near native speed in the browser, and using the proposed
system, the "split" method. 

\textbf{The Javascript code} has been implemented using the JSBN library
\cite{jsbn} which is the fastest in the mathematical open source libraries in
Javascript available.  Only three lines of code are needed to perform the
computation using JSBN. 

\textbf{The WebAssembly code} is ported from a C code base using the GMP
library. The GMP library and the final C code have been compiled using the
Emscripten framework~\cite{zakai2011emscripten}. Javascript Many optimizations
have been used such making only one call from Javascript to WebAssembly for all
computations and with optimizations flags turned on during the compilation. 

\textbf{The server code} has been implemented as a web server using the Golang
programming language \cite{golang}. The web server listens for incoming HTTP
POST requests from the client containing the base, exponent and modulo.  The web
server then performs the computation using a binding to the GMP library
\cite{gmplib} for good performance and sends back the results. The servers run
on the same machine locally without any induced latency. It is probably safe to
add an average 50ms of latency but this work does not do it since it does not
influence the results.The format of the communication is done through JSON with
hexadecimal encoding of the parameters.  For this experiment, only three web
servers have been deployed on the same machine. 

The results can be found in the Figure~\ref{fig:modexp:js} and in
Figure~\ref{fig:modexp:wasm}. The proposed system clearly outperforms Javascript
by an order of magnitude for any number of requested modular exponentations and
for any modulus sizes. Surprisingly, the WebAssembly's performance for a 1024
bit key with one or two exponentiations equals the ones from the "split" method. 
However, adding more exponentiations or using larger keys makes WebAssembly performs
worst than the "split" method, again with a difference noted by an order of
magnitude. Moreover, WebAssembly is still a recent technology that may not be
supported on many majors outdated browsers.

We also note that there are fluctuations of the "split" method with a 1024 bit
key size.  We explain those fluctuations due to the fact that the experiment is
ran over a single laptop while other processes are also active and it is very
hard to do micro-benchmarking accurately and reliably in Javascript due to the
interpreted nature of Javascript.  The relatively short time of computation
makes it difficult to get a reliable

\section{Discussions} \label{discussions}

We discuss in this section some open questions and some hints for future work. 

\begin{enumerate} 

    \item \textbf{Threat model:} One assumptions the system makes is that the
        server are honest-but-curious. If we want to relieve that assumption,
        the system needs a \textit{recovery} mechanism for the client to still
        be able to vote.  Some solutions may include the following: 

        \begin{itemize} 

            \item \textbf{Computing natively:} One solution can simply be to
                perform the computation using the native Javascript engine as a
                fallback mechanism. The client must be aware that the
                computation went wrong so he has an explicit reason to wait
                longer for the computation to finish.  

            \item \textbf{Invidividual verifiability:} Another solution can be
                to require each servers to provide a valid NIZK proof of
                exponentiation. In this case, this proof can be instantiated as
                a Schnorr signature using $r_i$ as the private key and $b^{r_i}$
                as the public key.  

        \end{itemize} 

\end{enumerate}

\section{Conclusion}

We presented a system decentralizing an expensive computation to lightweight
servers using the GMP computational library. The system shows a significant gain
in performance compared to a native Javascript implementation or a WebAssembly
compiled version and is therefor, a good target for the CHVote system.


\bibliographystyle{plain}
\bibliography{main}

\end{document}
